\subsubsection*{Coordinate systems}


  An important concept to understand when using EdaRenderer is the distinction between Cairo's ``device coordinates'' and ``user coordinates''. (Yes, go and look it up now).


  Before calling any libgedacairo drawing functions, Cairo context's transformation matrix must be setup so that the user coordinates match coordinates on the schematic or symbol page.



\subsubsection*{The EdaRenderer class}
 The EdaRenderer class is the main interface to the library. The EdaRenderer class uses information from libgeda OBJECT structures to draw various schematic elements, control grips, and cues.

  EdaRenderer is a GObject class, with several properties that control its behaviour: \begin{verbatim}

  "cairo-context"  [pointer]
      The `cairo_t' drawing context which the renderer draws to.

  "pango-context"  [pointer]
      The state of the Pango string-to-glyph itemiser.  If not explicitly set
      the renderer will automatically create one from the drawing context.

  "font-name"  [string]
      The font to use when drawing text.  By default, this is "Arial".

  "color-map"  [pointer]
      A GArray of libgeda `COLOR' structures which is used as the color
      map for drawing.  Note that the EdaRenderer does not make a copy
      of this array, and does not freethe array when the renderer is
      destroyed.

  "override-color"  [int] Index of a colour to force all drawing to
      take place with.  This can be used e.g. when highlighting a set
      of objects in a particular colour.  If "override-color" is set
      to -1, color override is disabled.

  "grip-size"  [double]
      Size to draw grips, in *user* coordinates.

  "render-flags"  [EdaRendererFlags]
      Flags that control various rendering options.

\end{verbatim}



  A EdaRendererFlags value can include any of these flags: \begin{verbatim}

  EDA_RENDERER_FLAG_HINTING
      Enable line hinting.  Set this flag if you are rendering to
      screen or to a raster image file, because it makes things look
      prettier.

  EDA_RENDERER_FLAG_PICTURE_OUTLINE
      Draw only the outlines of pictures, and not their contents.

  EDA_RENDERER_FLAG_TEXT_HIDDEN
      Draw text even if it is set to be invisible.

  EDA_RENDERER_FLAG_TEXT_OUTLINE
      Draw only the outlines of text objects.

  EDA_RENDERER_FLAG_TEXT_ORIGIN
      Draw small "X" markers indicating where text objects are
      anchored.

\end{verbatim}
 To actually draw things, EdaRenderer has three methods: \begin{verbatim}

  eda_renderer_draw()
      This does basic drawing of an OBJECT.

  eda_renderer_draw_grips()
      Draw grips at all of the control points of an OBJECT.

  eda_renderer_draw_cues()
      Draw cues for an object (e.g. at the unconnected ends of pins and
      nets, and at junctions in nets and buses).

\end{verbatim}
 Sometimes, it is useful to obtain the rendered bounding box of an OBJECT (for example, a text OBJECT's bounding box is determined by the font). \begin{verbatim}

  eda_renderer_get_user_bounds()
      Obtain the user bounds (which should be the same as world
      bounds) for an OBJECT.

\end{verbatim}
 These four methods all can be overridden from subclasses.
\subsubsection*{Lower-level drawing}
 Sometimes EdaRenderer is not enough! libgedacairo provides a bunch of functions for doing lower-level drawing.

  Several of these drawing functions take a `flags' argument. At the moment, the only flag that may be applied is `EDA\_CAIRO\_ENABLE\_HINTS'. If you have an EdaRenderer and you wish to obtain the same EdaCairoFlags value as used internally by the renderer, you can call eda\_renderer\_get\_cairo\_flags().


  The `center' variants of the drawing functions should be used to draw shapes that need to be symmetrically centred on a particular point. The `center\_width' arguments that these functions accept specify the width of the line that they are to be centred on, if applicable. The distinction is required in order to make sure line hinting takes place correctly.


  The available shape drawing functions are: \begin{verbatim}

  eda_cairo_line()
      Draw a straight line segment.

  eda_cairo_box(),  eda_cairo_center_box()
      Draw a rectangular box.

  eda_cairo_arc(), eda_cairo_center_arc()
      Draw a circular arc segment.

  eda_cairo_path()
      Draw a path made up of straight line and Bézier curve segments.

\end{verbatim}
 To actually do the drawing, there are two additional functions: \begin{verbatim}

  eda_cairo_set_source_color()
      Use a colour map index and a GArray of COLOR structures to set
      the current Cairo drawing colour.

  eda_cairo_stroke()
      Stroke the current Cairo path, with libgeda-compatible stroke
      parameters.

\end{verbatim}

