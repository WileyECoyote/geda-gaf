Libgedacairo(1)                 1.6.1-20190401                 Libgedacairo(1)



NAME
       Libgedacairo - Cairo-based Schematic and Symbol Renderer Library

DESCRIPTION
       The Libgedacairo library provides a renderer for schematics and symbols
       based on the Cairo vector graphics library and the Pango font  library.
       Data for rendering is loaded using libgeda.

       The  library  does  *not*  provide  a  rendering  widget  for programs,
       although Libgedacairo could be used to implement one.  The  library  is
       intended for more general rendering usage, e.g. to screen, to printers,
       or to various image formats.

CONFIGURATION
       Library operations are controlled by programmatically  setting  proper-
       ties  of  the  EdaRenderer  object, see below for more information. For
       details on setting up linkage see libgedacairo(3)

ENVIRONMENT
       Libgedacairo does not recognize any environment  variables.  Operations
       are  controlled  by  programmatically setting properties of the EdaRen-
       derer object. The Cairo library does  recognize  a  limited  number  of
       environment variables :

          CAIRO_GL_COMPOSITOR

          CAIRO_GALLIUM_LIBDIR

          CAIRO_GALLIUM_FORCE

       See the Cairo library documentation for information on the use of these
       environment variables.

USAGE
          COORDINATE SYSTEMS
              An important concept to understand when using an EdaRenderer  is
              the  distinction  between Cairo's "device coordinates" and "user
              coordinates". (Yes, go and look it up now).

              Before calling any Libgedacairo  drawing  functions,  the  Cairo
              context's transformation matrix must be set so that user coordi-
              nates match coordinates on the schematic or symbol page.

          EDARENDERER CLASS
              The EdaRenderer class is the main interface to the library.  The
              EdaRenderer  class  uses  information from libgeda OBJECT struc-
              tures to draw various schematic  elements,  control  grips,  and
              cues.

              An  EdaRenderer  is a GObject with the following properties that
              control behaviour:

                "cairo-context"  [pointer]
                    The Cairo drawing context which the renderer draws to.

                "pango-context"  [pointer]
                    The state of the Pango string-to-glyph itemizer. The  ren-
              derer
                    will  automatically create one from the drawing context if
              not
                    explicitly set.

                "font-name"  [string]
                    The font to use when drawing text.  By  default,  this  is
              "Arial".

                "color-map"  [pointer]
                    A  GArray  of  libgeda "COLOR" structures which is used as
              the
                    color map for drawing. Note that the EdaRenderer does  not
              make
                    a copy of this array, and doesn't free it when the EdaRen-
              derer
                    is destroyed.

                "override-color"  [int]
                    Index of a color to force all drawing to take place  with.
              This
                    can  be  used e.g. when highlighting a set of objects in a
              particular
                    color. If "override-color" is set to -1, color override is
              disabled.

                "junction-color"  [GDK_TYPE_COLOR]
                    GDK color to use when rendering Junctions.

                "junction-size"  [interger]
                    Size to draw junction cue points.

                "net-endpoint-color"  [GDK_TYPE_COLOR]
                    GDK color to use when rendering Net and Pin endpoints.

                "text-marker-color"  [GDK_TYPE_COLOR]
                    GDK color to use when rendering text markers.

                "text-marker-size"  [interger]
                    Size to draw text markers.

                "text-marker-threshold"  [double]
                    The threshold to draw text markers.

                "draw-grip"  [Boolean]
                    Controls if grips should be drawn.

                "grip-size"  [double]
                    Size to draw grips, in *user* coordinates.

                "grips-stroke"  [GDK_TYPE_COLOR]
                    GDK color to use when rendering strokes for grips.

                "grips-fill"  [GDK_TYPE_COLOR]
                    GDK color to use when rendering background of grips.

                "circle-grip-quadrant"  [GDK_TYPE_COLOR]
                    Controls where grips are drawn on circles.

                "render-flags"  [EdaRendererFlags]
                    Flags that control various rendering options.

              A EdaRendererFlags value can include any of these flags:

                EDA_RENDERER_FLAG_HINTING
                    Enable  line  hinting.  Set  this flag when rendering to a
              screen
                    or raster image file, because it makes things  look  pret-
              tier.
                EDA_RENDERER_FLAG_PICTURE_OUTLINE
                    Draw  only  the  outlines  of pictures, and not their con-
              tents.
                EDA_RENDERER_FLAG_TEXT_HIDDEN
                    Draw text even if it is set to be invisible.
                EDA_RENDERER_FLAG_TEXT_OUTLINE
                    Draw only the outlines of text objects.
                EDA_RENDERER_FLAG_TEXT_ORIGIN
                    Draw small "X" markers indicating where text objects are
                    anchored.

              To actually draw shapes or objects, EdaRenderer has three  meth-
              ods:

                eda_renderer_draw()
                    This does basic drawing of an OBJECT.

                eda_renderer_draw_grips()
                    Draw grips at all of the control points of an OBJECT.

                eda_renderer_draw_cues()
                    Draw  cues  for an object (e.g. at the unconnected ends of
              pins and
                    nets, and at junctions in nets and buses).

              Sometimes, it is useful to obtain the rendered bounding  box  of
              an  OBJECT  (for example, a text OBJECT's bounding box is deter-
              mined by the font).

                eda_renderer_get_user_bounds()
                    Obtain the user bounds (which should be the same as world
                    bounds) for an OBJECT.

              These methods all can be overridden from subclasses.

DOCUMENTATION
          SOURCE CODE
              The gEDA-gaf suite utilizes Doxygen for generating source  docu-
              mentation.  By  default  this documentation is not generated and
              must be enabled via  the  configuration  option  --with-doxygen.
              When enabled, the documentation is then built in the source tree
              using "make doxygen" from the top of the source. To only  gener-
              ate  source  documentation  for  Libgedacairo use "make doxygen"
              from within the top libgeda directory. Note that  the  configure
              script  can  also set the configuration to allows generating the
              source documentation out of the source tree  using  the  --with-
              doxygen-out    option.   For   example   using   --with-doxygen-
              out=/tmp/geda-gaf will cause the resulting documentation  to  be
              generated in the "/tmp/geda-gaf" directory. The target directory
              will created if it does not already exist. It is  not  necessary
              to  use both doxygen options, using --with-doxygen-out automati-
              cally enables --with-doxygen.  The Doxygen configuration is  set
              to  produce both HTML and latex. Other formats are possible, see
              the Doxygen documentation for details.

AUTHORS
       See the "AUTHORS" file included with this program.

COPYRIGHT
       Copyright (C) 2019 gEDA Contributors. License GPLv2+: GNU GPL
       version 2 or later. Please see the `COPYING' file included with this
       program for full details.

       This is free software: you are free to change and redistribute it.
       There is NO WARRANTY, to the extent permitted by law.

SEE ALSO
       libgedacairo(3), libgeda(1), libgedathon(1), libgedauio(1) gnetlist(1),
       gschem(1), gsymcheck(1)

       http://wiki.geda-project.org/geda:documentation.



gEDA Project                     April 1, 2019                 Libgedacairo(1)
