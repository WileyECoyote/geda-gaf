Libgeda(1)                      51.0.0-20151031                     Libgeda(1)



NAME
       Libgeda - The GPL Electronic Design Automation core library

DESCRIPTION
       Libgeda is a set of library routines used by applications to manipulate
       data for electronic schematics and symbols, primarily in the gEDA suite
       of applications.  Libgeda provides utility functions for creating, mod-
       ifying, parsing and traversing gEDA data, handles all disk  i/o  opera-
       tions  for  gEDA  applications,  as well as providing an API for Scheme
       based programs and routines for evaluation of  Scheme  expressions  and
       files  for  gEDA applications.  Libgeda also provides a limited collec-
       tion of other common routines like string manipulators, a configuration
       system,  and math functions useful in gEDA applications. The purpose of
       the this manual page is to explains  user  oriented  aspects  of  using
       Libgeda  ,  specifically, configuring the library and providing details
       on where to find documentation on other aspect of Libgeda such  as  the
       Scheme API.

CONFIGURATION
       Library  operations  are controlled parameter variables and some of the
       key parameters are exposed so that  installations  can  be  customized.
       These  key  parameters  are  associated  with  "keywords" in text based
       files. The files are sometimes referred to as  initialization  or  "rc"
       files  and in the case of Libgeda these files are really Scheme scripts
       that are either executed or sourced.  The  particular dialect of Scheme
       used  by  Libgeda  is GNU's Guile. While it is required for Guile to be
       installed on the host system for Libgeda to run, it is not  a  require-
       ment  that  gEDA  applications  initialize  or  even  be running Guile.
       Libgeda does not initiate the loading "rc" files. Instead, when  appli-
       cation  request  that  a initialization file be processed Libgeda first
       checks for "system-gafrc" in either system  configuration  directories,
       i.e.  "/etc/gEDA",  "~/etc",  or the root of the installation directory
       and precesses these files if they exist. Normally, only one  such  file
       is  used  as a system level "master" configuration file. Following pro-
       cessing of the "system-gafrc" files Libgeda searches the current direc-
       tory  for  local  "gafrc"  configurations  files.  The search continues
       upward in the directory structure until either a  file  with  the  name
       "gafrc"  is found or the root of the drive is reached without discover-
       ing such a file. If discovered, the "gafrc" file  is  processed,  which
       may  over-ride  configuration  parameters established in preceding ini-
       tialization files, and this allows users to customize Libgeda on a user
       and or on project bases. Local initialization files are "remembered" so
       that when applications load other schematics or symbols in the same  or
       subordinate  directory  structure the initialization files are not pro-
       cessed a second time for a given  session.  The  application  requested
       initialization files are then processed in a likewise manner, for exam-
       ple system-gschemrc.  Initialization files for Libgeda should not  ref-
       erence  application  specific keywords or functions. Since the initial-
       ization files are really Scheme scripts the possibilities for  configu-
       ration customization is essentially unlimited. The basic "system-gafrc"
       file installed with Libgeda contains details on each keyword.  For  the
       most part, the capability to perform customization and modifications to
       parameters are provided for advanced users, allowing users to deal with
       complex installations and maximizes flexibility of the suite. For exam-
       ple, system administrators may want to  change  where  logs  files  are
       written,  while  ordinary user may desire to disable automatic backups.
       Project managers might want to modify the list of  promoted  attributes
       for  a  particular  project.  Users  should backup initialization files
       before making any modifications to the files.

ENVIRONMENT
       The following environment variables are recognized by Libgeda :

          GEDADATA
              Specifies the search directory for Scheme and bitmap files.  The
              default  is  compilation  dependent based on the location of the
              data directory and can be set using the -datarootdir  configura-
              tion  option. Normally the location is "prefix/share/gEDA" where
              "prefix" is the root of the installation  build.   The  GEDADATA
              environment setting over-rides the default location.

          GEDADATARC
              Specifies the search directory for rc files. The default is com-
              pilation dependent based on the location of the  data  directory
              and  can  be  set  using  --with-rcdir configuration option. The
              default is the same as GEDADATA.  Specifying a GEDADATARD  envi-
              ronment setting over-rides the default location.

          GEDALOGS
              Specifies  the  directory for log files. The default is compila-
              tion dependent based on the location of the  user  configuration
              files  and  can  be controlled using --with-logdir configuration
              option. The GEDALOGS environment setting over-rides the  default
              location.

DOCUMENTATION
          SOURCE CODE
              The  gEDA-gaf suite utilizes Doxygen for generating source docu-
              mentation. By default this documentation is  not  generated  and
              must  be  enabled  via  the configuration option --with-doxygen.
              When enabled, the documentation is then built in the source tree
              using  "make doxygen" from the top of the source. To only gener-
              ate source documentation for Libgeda  use  "make  doxygen"  from
              within the top libgeda directory. Note that the configure script
              can also set the configuration to allows generating  the  source
              documentation  out  of the source tree using the --with-doxygen-
              out option. For example  using  --with-doxygen-out=/tmp/geda-gaf
              will  cause  the  resulting documentation to be generated in the
              "/tmp/geda-gaf" directory. The target directory will created  if
              it does not already exist. It is not necessary to use both doxy-
              gen  options,  using  --with-doxygen-out  automatically  enables
              --with-doxygen.   The  Doxygen  configuration  is set to produce
              both HTML and latex. Other formats are possible, see the Doxygen
              documentation for details.

AUTHORS
       See the "AUTHORS" file included with this program.

COPYRIGHT
       Copyright (C) 2015 gEDA Contributors. License GPLv2+: GNU GPL
       version 2 or later. Please see the `COPYING' file included with this
       program for full details.

       This is free software: you are free to change and redistribute it.
       There is NO WARRANTY, to the extent permitted by law.

SEE ALSO
       libgedacairo(1),  libgedathon(1), libgedauio(1) gnetlist(1), gschem(1),
       gsymcheck(1)

       http://wiki.geda-project.org/geda:documentation.



gEDA Project                                                        Libgeda(1)
