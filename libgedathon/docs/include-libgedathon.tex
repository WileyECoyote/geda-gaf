 Inclusion of the geda module. 

  By default the geda module is installed to the local Python module directory. On Linux machines this might be something like ``/usr/local/lib/python2.7/site-packages/geda''. To import the module, the module must be located within the search path of the interpreter. If the module was installed to a non-standard location, such as dist-packages, the importer might not find the geda module. The Python search path can be set using various methods. A common method is to use the environment variable PYTHONPATH, which may contain a colon-separated list of directories. The PYTHONPATH variable can be appended using something like:\\ 
\\ 
  export PYTHONPATH=\$PYTHONPATH:THE/PATH/TO/GEDA\\ 
\\ 
 Other methods can be used to modify the Python search path, such as using sys.path.append('/path/to/search') or site.addsitedir. See the Python documentation for more information on using such methods. 
\\ 
  To include the geda module use: \\ 
\\ 
from geda import geda\\ 
\\ 
  The high level functions can be imported using: \\ 
\\ 
from geda.functions import *\\ 
\\ 
  The high level functions provide an interface to the base module with minimal type checking and no defaulting. The base module provides defaults for all optional parameters. 
\\ 
  It is also convenience to include all of the constants separately with: \\ 
\\ 
from geda.constants import *\\ 
\\ 
  It is not required to directly import the high level functions and constants separately, but doing so is more convenience. Submodules could be referenced using statements like: \\ 
\\ 
geda.functions.AddCapacitor(schematic, 6700, 8600, ``20nF'')\\ 
 or\\ 
sheet1 = geda.constants.SHOW\_NAME\_VALUE\\ 
 or even\\ 
sheet1 = geda.SHOW\_NAME\_VALUE\\ 
