 Using the geda module. 

  Libgedathon is automatically initialized by the geda module when the module is imported. All functions and methods are available immediately. LibgEDA is initialized by Libgedathon, including the processing of the system-gafrc file. There are no specific Python configuration files. While LibgEDA is a shared static library, Libgedathon is dynamically linked, even on Linux machines. The full symbol library will be available from from the installed library path. This path is append-able using the Libgedathon API. There is also utility functions to create local ``gafrc'' files, which are typically used to append the library search path. 


  In general, programs will start either by opening existing files, schematic or symbols, or creating new files. As a quick example, these operations could be performed using:

schematic = geda.new\_page(filename)\\ 
 or\\ 
sheet1 = geda.open\_page(path + ``sheet1'')


  The returned objects represent the page. This object must be supplied to most methods, such all methods in the Creation group, as Libgedathon does not currently support the ``current page'' scheme supported by LibgEDA. With Libgedathon the page must always be specified. 


  Pages should be saved before closing using either a module level function or the objects methods. Using the page names from the preceding example, the page could be saved and closed using either of these methods:

geda.save\_page(schematic)\\ 
 or\\ 
sheet1.save()


  Files and can also be saved when closing by suppling an optional integer flag:

geda.close\_page(sheet1, True)

  Objects are added or removed from pages using either the module API or the page object methods. In either case the program must have a reference to the object first. Methods in the Creation group create new objects based on supplied parameters, as in this example:

pinlabel = geda.new\_attrib(``pinlabel'', ``2'', x, y1, INVISIBLE, SHOW\_VALUE, LOWER\_RIGHT)

  Note that the preceding example uses three constants defined in the constants submodule. The Text object pinlabel can now be added to another object. In the following example the label is being added to a GedaPin object:

geda.add\_object(pin, pinlabel)

  which could have also been accomplished using the Pin object's add method:

pin.add(pinlabel)


  The geda module includes a collection of high-level functions in a submodule, and these functions automate tasks such as creating and adding objects by combining operations. The following example shows how a capacitor symbol can be created and inserted into a page in a single step using a high-level function:

AddCapacitor(schematic, 6700, 8600, ``20nF'')


  Using high-level functions does not significantly impact performance, Libgedathon was designed to be efficient, performing operations using ``C'' code rather than using the Python interpreter. The high-level functions typically use the geda module not the Object's methods. Functions in the geda module generally use Python for type checking as the arguments passed to the module originate in Python code. Object methods tend to perform data validation at a lower level. High-level functions also except keyword arguments.


  The high-level Adder functions, such as the one in the previous example, AddCapacitor, use default symbols. The default symbol names are read-writable and can be changed on an as needed basis using the corresponding setter functions, as show in these examples:

DefaultCapacitorSymbol(``capacitor-py'')
\\ 
DefaultOpAmpSymbol(``dual-opamp-py'')


  The default settings are not saved between sessions. These examples actually show the ``real'' default symbol name, but could name any valid symbol name. Typically a program would set the symbol name and then insert, or at least create, all symbols of that type before changing to a different symbol. To get the name of the current default symbol use the value returned from the same function without an argument:

current\_titleblock = DefaultTitleblockSymbol()


  Default symbols are only applicable to basic component types such as capacitors, resistors, and transistors. Complex types must supply a valid system name. Complexes can be created and added using either the geda module or functions in the high-level submodule as shown in the following examples:

titleblock = geda.new\_complex(``title-B'', 1000, 1000)\\ 
 titleblock.locked = True\\ 
 geda.add\_object(sheet1, titleblock)\\ 
 or\\ 
tb = AddComponent(sheet1, titleblock, , 1000, 1000)


  In the last example the symbol name was in a variable named titleblock, but could contain any valid symbol name. The example illustrates the convenience of using the high-level function. The variable ``tb'' holds a reference to the object, which might not be needed for a titleblock, in which case, the previous example could have been reduced to:

AddComponent(sheet1, titleblock, , 1000, 1000)


  In this case, since there is no reference to the object, the Python version of the object will be destroyed immediately, but the gobject version still exist in LibgEDA and is referenced by the page. When the geda module method was used, Libgedathon maintained a reference to the Python object but released the reference when the object was added to the Page, in the example above using geda.add\_object. The program could have saved a reference using the module method just as easily by referencing the object returned from geda.add\_object, i.e. tb=geda.add\_object. 


  Note that in the previous examples for AddComponent, the third parameter, which is for the reference designator, was left blank. This is because titleblocks are not normally assigned designators but the positional syntax was used as oppose to using keywords. To illustrate the difference, the following example using mixed-mode syntax is equivalent to the preceding example:

AddComponent(sheet1, titleblock, ``x''=1000, ``y''=1000)


  The high-level Adder functions also feature auto-referencing and except either strings or integers for the reference designator arguments. The following statements would all add a 560 Ohms resistor and assign ``R1'' as the reference designator:

AddResistor(amplifier, 10400, 7600, ``560'', R1)
\\ 
AddResistor(amplifier, 10400, 7600, ``560'', 1)
\\ 
AddResistor(amplifier, 10400, 7600, ``560'', ``1'')


  If the resistor really was the first resistor to be added to the amplifier object then the following two statements would also add a resistor and assign ``R1'' as the reference designator: 
AddResistor(amplifier, 10400, 7600, ``560'')
\\ 
AddResistor(amplifier, 10400, 7600, ``560'', 0)


  In the later case, the 0 means explicitly to use auto-referencing. Auto-referencing can be overridden for any component at any time by specifying the reference designator as was the case in the first statement. If the second statement had been the 4th or 5th, then ``R2'' would be assigned as the reference designator because the auto referencing systems continues to track designations even when being overridden. Note that the auto-referencing system is intended to be fully automatic and there are no configurable options, the system is either used or it is not used. 


  Libgedathon and the geda extension module also provide methods for obtaining and manipulating attributes, such in as in this slightly more complex example: \begin{verbatim}
	pin = geda.new_pin(500, 600, 500, 800, 1, 8, "V+", PIN_ELECT_PWR)
	sequence = sequence + 1
	pin.sequence = sequence
	geda.add_object(opamp, pin)
	butes = geda.get_attribs(pin)
	for attrib in butes:
		if attrib.name() == "pinnumber":
		    attrib.y = attrib.y + 100
		else:
		    if attrib.name() == "pinlabel":
		        attrib.y     = attrib.y + 100
		        attrib.x     = attrib.x + 250
		        attrib.angle = 0
\end{verbatim}

  A complete example is provided with the source code, the script is named ``lpbf.py''. lpbf uses three other scripts, capacitor.py, dual-opamp.py, and resistor.py to create three custom local symbols, and then creates a complete schematic design for a 4th-order low-pass Butterworth filter with the default title-block and the three local symbols using Python. 

  Detailed information on gEDA Python API properties, methods and functions can be obtain from the doxygen generated documents. Basic syntax and descriptions can also be obtained for all functions and methods via the Python help interface. 
