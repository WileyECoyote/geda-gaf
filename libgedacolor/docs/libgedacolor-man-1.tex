libgEDAColor(1)                    -20160214                   libgEDAColor(1)



NAME
       libgEDAColor - The GPL Electronic Design Automation Color Library

DESCRIPTION
       libgEDAColor  is an extension library used in the gEDA suite to coordi-
       nate the mapping of color codes used in gEDA data files to actual  col-
       ors  used  for  output devices based on user configuration preferences.
       libgEDAColor provides functions for parsing gEDA scheme color-map  data
       and  API  functions  for  applications to access color information. The
       this manual page explains user aspects of using libgEDAColor , specifi-
       cally,  configuring  the library and providing details on where to find
       documentation on other aspect of libgEDAColor such as  the  application
       API.

CONFIGURATION
       Library  operations  are controlled parameter variables and some of the
       key parameters are exposed so that  installations  can  be  customized.
       These  key  parameters  are  associated  with  "keywords" in text based
       files. The files are sometimes referred to as  initialization  or  "rc"
       files  and  in  the  case of libgEDAColor these files are really Scheme
       scripts that are either executed or sourced.  The   particular  dialect
       of Scheme used by libgEDAColor is GNU's Guile. While it is required for
       Guile to be installed on the host system for libgEDAColor to run, it is
       not  a requirement that gEDA applications initialize or even be running
       Guile.  libgEDAColor does not initiate the loading "rc" files. Instead,
       when  application  request  that  a  initialization  file  be processed
       libgEDAColor first checks for "system-gafrc" in either system  configu-
       ration  directories,  i.e.  "/etc/gEDA",  "~/etc",  or  the root of the
       installation directory and precesses these files if  they  exist.  Nor-
       mally, only one such file is used as a system level "master" configura-
       tion file. Following processing of the  "system-gafrc"  files  libgEDA-
       Color  searches  the current directory for local "gafrc" configurations
       files. The search continues upward in  the  directory  structure  until
       either  a  file with the name "gafrc" is found or the root of the drive
       is reached without discovering such a file. If discovered, the  "gafrc"
       file  is processed, which may over-ride configuration parameters estab-
       lished in preceding initialization files, and this allows users to cus-
       tomize  libgEDAColor  on a user and or on project bases. Local initial-
       ization files are "remembered" so that  when  applications  load  other
       schematics  or  symbols  in the same or subordinate directory structure
       the initialization files are not processed a second time  for  a  given
       session.  The  application requested initialization files are then pro-
       cessed in a likewise manner, for example system-gschemrc.   Initializa-
       tion  files  for libgEDAColor should not reference application specific
       keywords or functions. Since the initialization files are really Scheme
       scripts  the  possibilities  for  configuration customization is essen-
       tially unlimited. The basic "system-gafrc" file installed with libgEDA-
       Color contains details on each keyword. For the most part, the capabil-
       ity to perform customization and modifications to parameters  are  pro-
       vided for advanced users, allowing users to deal with complex installa-
       tions and maximizes flexibility  of  the  suite.  For  example,  system
       administrators  may  want to change where logs files are written, while
       ordinary user may desire to disable automatic backups. Project managers
       might  want  to modify the list of promoted attributes for a particular
       project. Users should backup initialization  files  before  making  any
       modifications to the files.

ENVIRONMENT
       The following environment variables are recognized by libgEDAColor :

          GEDADATA
              Specifies  the search directory for Scheme and bitmap files. The
              default is compilation dependent based on the  location  of  the
              data  directory and can be set using the -datarootdir configura-
              tion option. Normally the location is "prefix/share/gEDA"  where
              "prefix"  is  the  root of the installation build.  The GEDADATA
              environment setting over-rides the default location.

          GEDADATARC
              Specifies the search directory for rc files. The default is com-
              pilation  dependent  based on the location of the data directory
              and can be set  using  --with-rcdir  configuration  option.  The
              default  is the same as GEDADATA.  Specifying a GEDADATARD envi-
              ronment setting over-rides the default location.

          GEDALOGS
              Specifies the directory for log files. The default  is  compila-
              tion  dependent  based on the location of the user configuration
              files and can be controlled  using  --with-logdir  configuration
              option.  The GEDALOGS environment setting over-rides the default
              location.

DOCUMENTATION
          SOURCE CODE
              The gEDA-gaf suite utilizes Doxygen for generating source  docu-
              mentation.  By  default  this documentation is not generated and
              must be enabled via  the  configuration  option  --with-doxygen.
              When enabled, the documentation is then built in the source tree
              using "make doxygen" from the top of the source. To only  gener-
              ate  source  documentation  for  libgEDAColor use "make doxygen"
              from within the top libgEDAColor directory. Note that  the  con-
              figure  script can also set the configuration to allows generat-
              ing the source documentation out of the source  tree  using  the
              --with-doxygen-out  option.  For  example  using --with-doxygen-
              out=/tmp/geda-gaf will cause the resulting documentation  to  be
              generated in the "/tmp/geda-gaf" directory. The target directory
              will created if it does not already exist. It is  not  necessary
              to  use both doxygen options, using --with-doxygen-out automati-
              cally enables --with-doxygen.  The Doxygen configuration is  set
              to  produce both HTML and latex. Other formats are possible, see
              the Doxygen documentation for details.

AUTHORS
       See the "AUTHORS" file included with this program.

COPYRIGHT
       Copyright (C) 2016 gEDA Contributors. License GPLv2+: GNU GPL
       version 2 or later. Please see the `COPYING' file included with this
       program for full details.

       This is free software: you are free to change and redistribute it.
       There is NO WARRANTY, to the extent permitted by law.

SEE ALSO
       libgEDAColorcairo(1),      libgEDAColorthon(1),      libgEDAColoruio(1)
       gnetlist(1), gschem(1), gsymcheck(1)

       http://wiki.geda-project.org/geda:documentation.



gEDA Project                                                   libgEDAColor(1)
